\section{Multiplicative Functions}

Multiplicative functions provide enlightening examples of symmetric shadow structures, and symmetric shadows provide algebaric structure to classes of multiplicative function. They also provide an isomorphism with Tannakian symbols, allowing us to compute multiplicative function arithmetic symbolically. This can be used for automated reasoning about multiplicative functions, as in \ref{???}. 

\subsection{The main structure}

Let $k$ be a field of characteristic $0$. We define the set $Mult(k)$ of all multiplicative functions $\N \to k$. We form a bijection with $\Upsilon_{k^{\P}}$, and define a symmetric shadow structure on $Mult(k)$ by this bijection. We make several important remarks on the structure we obtain.

 NOTE!!!!!!! This defines the box structure!

\begin{definition}
    Let the set $Mult(k)$ be the set of all multiplicative functions $\N \to k$, that is all functions $f$ such that $f(ab) = f(a)f(b)$ whenever $a$ and $b$ are coprime.
\end{definition}

\begin{propdef}
    We define a bijection $\phi : Mult(k) \to \Upsilon_{k^{\P}}$ by taking $f$ to $a_i = p \mapsto f(p^i)$. 
\end{propdef}

\begin{proof}
    The data of a multiplicative function is exactly its values at $p^e$ for $p$ prime and $e \ge 1$, and these values are chosen freely. Hence all the data of a the multiplicative function is transferred to the zeta type structure, which itself is completely determined by its $a_i$ which each is a function from the primes to $k$. The values of these functions are also chosen freely. More formally, if $a_i = p \to a_{i, p}$ then the inverse of the transformation is just taking the multiplicative function $f(p^e) = a_{e, p}$. 
\end{proof}

\begin{definition}
    We define the symmetric shadow structure $Mult(k)$ by 
\end{definition}

\begin{theorem}
    For the structure defined 
\end{theorem}
