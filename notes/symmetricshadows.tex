\section{Symmetric Shadows}

In general, a zeta type is thought of as some kind of numerical manifestation of a higher object, or a \emph{shadow}, if you will. With this intuition in mind it makes sense to form axioms for structures that give rise to zeta types, which we informally call zeta type formulas.
 
 The trivial approach would be to define this as a function $f : A \to \Upsilon_k$, where $A$ would be the set of higher objects of a particular kind, and $k$ is a field containing the kind of numbers that the objects give rise to (for instance the rationals containing integers). 
 
 However, this ignores all structural aspects of zeta type formulas, which at the very least in some sense should respect addition and product (a central philosophy of zeta types is that they should simplify complicated structures faithfully). This can be encapsulated by making $f$ a ring-homomorphism and enriching $A$ with some kind of addition and multiplication. 
 
 Further, something truly beatiful happens if we require that the zeta types formula also respect some notion of symmetry. $\lambda$-rings serve as precise algebraic notion of symmetry, and hence we make $f$ a $\lambda$-homomorphism and define suitable $\lambda$-operations on $A$. What now happens is that the formula $f$ collapses into a single ring-homomorphism $\upsilon : A \to k$, which is both sufficient and required for the existence of a unique formula $f : A \to \Upsilon_k$.

\subsection{Central definition and the Umbra}

\subsection{Morphisms of Symmetric Shadows}

\subsection{The Nebula}

\subsection{The Image of a Symmetric Shadow}

\subsection{Some Simple Symmetric Shadows}

\subsection{The Rational Shadow}

\subsection{Tannakian Shadows}
