\section{Symmetric Shadows}

In general, a zeta type is thought of as some kind of numerical manifestation of a higher object, or a \emph{shadow}, if you will. With this intuition in mind it makes sense to form axioms for structures that give rise to zeta types, which we informally call zeta type formulas.
 
 The trivial approach would be to define this as a function $f : A \to \Upsilon_k$, where $A$ would be the set of higher objects of a particular kind, and $k$ is a field containing the kind of numbers that the objects give rise to (for instance the rationals containing integers). 
 
 However, this ignores all structural aspects of zeta type formulas, which at the very least in some sense should respect addition and product (a central philosophy of zeta types is that they should simplify complicated structures faithfully). This can be encapsulated by making $f$ a ring-homomorphism and enriching $A$ with some kind of addition and multiplication. 
 
 Further, something truly beatiful happens if we require that the zeta types formula also respect some notion of symmetry. $\lambda$-rings serve as precise algebraic notion of symmetry, and hence we make $f$ a $\lambda$-homomorphism and define suitable $\lambda$-operations on $A$. What now happens is that the formula $f$ collapses into a single ring-homomorphism $\upsilon : A \to k$, which is both sufficient and required for the existence of a unique formula $f : A \to \Upsilon_k$.

\subsection{Central definition and the Umbra}
We begin by defining the central concepts of symmetric shadows and some terminology, as well as establishing the relation between zeta type formulas and ring-homomorphisms to the base field. 


\begin{definition}
  A \emph{(local) symmetric shadow} is a triple $(R, k, \upsilon)$ where $R$ is a $\lambda$-ring called the \emph{symmetry realm}, $k$ is a field called the \emph{shadow field}, and $\upsilon: R \to k$ is a ring-homomorphism called the \emph{umbra}. 
  \begin{enumerate}
    \item An \emph{element} of the symmetric shadow refers to an element of the symmetry realm. 
    \item A symmetric shadow \emph{over} $k$ has $k$ as shadow field, and one \emph{under} $R$ has $R$ as symmetry realm.
  \end{enumerate}
\end{definition}

\begin{propdef}
  Given a symmetric shadow $\Gamma = (R, k, \upsilon)$ we may define a $\lambda$-homomorphism $f_\Gamma : R \to \Upsilon_k$ by $f_\Gamma(z) : n \mapsto \upsilon(\psi^n(z))$, where $\psi^n$ is the $n$-th Adams operation. 
\end{propdef}

\begin{propdef}
  Given a $\lambda$-homomorphism $f : R \to \Upsilon_k$ we may define a symmetric shadow $\Gamma_f = (R, k, \upsilon_k \circ f$.
\end{propdef}

%TODO: Categorify, and write about "and even naturally" 
\begin{theorem}
  These two notions commute, in that for any $\lambda$-homomorphism $f : R \to \Upsilon_k$, we have $f_{\Gamma_f} = f$ and for any symmetric shadow $\Gamma$ we have $\Gamma_{f_\Gamma} = \Gamma$. informally, this means that a zeta type formula is essentially (and even naturally, as we will see later) the same thing as a symmetric shadow.
\end{theorem}

\begin{proof}
  We begin with $\Gamma_{f_\Gamma} = \Gamma$. Say $\Gamma$ has the umbra $\upsilon$. It is simple to show that $\upsilon_k \circ f_\Gamma = \upsilon$, as $\upsilon_k(f_\Gamma(z)) = \upsilon_k(n \mapsto \upsilon(\psi^n(z))) = \upsilon(\psi^1(z)) = \upsilon(z)$. 
  
  For the first part, given a formula $f$, we want to show $f_{\Gamma_f} = f$. We begin by expanding the left hand side into $f_{\Gamma_f}(z) : n \mapsto \upsilon_k(f(\psi^n(z)))$, which must equal $f(z)$ for all $z$. By $f$ being a homomorphism, we can rewrite to $n \mapsto \upsilon_k(\psi^n(f(z)))$. Now notice that in $\Upsilon_k$, it holds that $\upsilon_k(\psi^n(l \mapsto a_l)) = \upsilon_k(l \mapsto a_{ln}) = a_n$. As such, we get $n \mapsto \upsilon_k(\psi^n(f(z))) = f(z)$.
\end{proof}

\subsection{Morphisms of Symmetric Shadows}
The notion of $f_\Gamma$ can be slightly cleaned up by defining morphisms between symmetric shadows. We also establish some more terminology, and reintroduce $\Upsilon_k$ as a symmetric shadow itself. 


\begin{definition}
  We define a morphism of symmetric shadows from $(R, k, \upsilon)$ to $(R', k', \upsilon')$ as a pair $(f, g)$ with $f : R \to R'$ a $\lambda$-ring homomorphism and $g : k \to k'$ a ring-homomorphism such that $\upsilon'(f(x)) = g(\upsilon(x))$. 
  \begin{enumerate}
    \item A morphism of symmetric shadows may be called an \emph{umbral morphism}. 
    \item Further, given a pair $(f, g)$, we call $f$ the \emph{symmetry part} and $g$ the \emph{shadow part}. 
    \item When the shadow part is the identity-morphism, we call the morphism \emph{anchored} (to $k$). 
    \item Composition of umbral morphisms is given by composition partwise. 
    \item A morphism is \emph{injective} when both parts are also injective. Notice however that morphisms of fields are always injective, so the symmetry part is suffient.
  \end{enumerate}
\end{definition}

\begin{remark}
  The category of symmetric shadows is clearly a comma-category
\end{remark}

\begin{propdef}
  We define the univeral symmetric shadow $\Upsilon_k = (k^{\N}, k, \upsilon_k : f \mapsto f(1))$ where $R^{\N}$ is equipped with pointwise addition and multiplication and Adams operations corresponding to compression, meaning $\psi^k(f) : n \mapsto f(nk)$.
\end{propdef}

\begin{propdef}
  For each symmetric shadow $\Gamma = (R, k, \upsilon)$ there is an anchored umbral morphism $\Psi_{\Gamma} : \Gamma \to \Upsilon_k$. The symmetry part is defined by $f_\Gamma$. We call $\Psi_{\Gamma}(r)$ the $\psi$-shadows of $r$.
\end{propdef}

\begin{propdef} \label{propdef:upsilonUniversalDiagram}
  For each morphism $\phi : \Gamma \to \Gamma'$ we get a commuting diagram:
  \begin{center}
    \begin{tikzcd}
      \Gamma \arrow[d, "\Psi_\Gamma"] \arrow[r, "\phi"] & \Gamma' \arrow[d, "\Psi_{\Gamma'}"] \\
      \Upsilon_k \arrow[r, "\Upsilon_{g}"] & \Upsilon_{k'}
    \end{tikzcd}
  \end{center}
  where $g : k \to k'$ is the shadow part of $\phi$ and $\Upsilon_{g}$ is defined as the morphism induced from applying the $g$ to each point in the symmetry-realm of $\Upsilon_R$ and of course as well to its shadow domain.
\end{propdef}

\begin{proof}
  Say $\phi = (f, g)$. We need to show that $\Upsilon_{g} \circ \Psi_\Gamma = \Psi_{\Gamma'} \circ \phi$. The shadow part is given by $g \circ id = id \circ g$, which is of course true, and the symmetric part is given by $g^{\N} \circ (r \mapsto k \mapsto \upsilon(\psi^k(r))) = (r \mapsto k \mapsto \upsilon'(\psi^k(r))) \circ f$. If we type this out using a single element $r$ and $k$, we get $g(\upsilon(\psi^k(r))) = \upsilon'(\psi^k(f(r)))) = \upsilon'(f(\psi^k(r)))$. This is true because $g(\upsilon(x)) = \upsilon'(f(x))$ is an axiom of umbral morphisms. \qedhere
\end{proof}

\subsection{The Nebula}
An ideal situation for zeta type formulas is that they should be faithful, ie. different objects should have different zeta types. We introduce the nebula to measure just how faithful a symmetric shadow is. We also make central observations that can be used to compute the nebula, and show some of the interesting consequences of faithfulness. 


\begin{definition}
  We define the \emph{nebula} of $\Gamma$, $N(\Gamma)$, as the kernel of the symmetric part of $\Psi_\Gamma$. This is a $\lambda$-ideal. We call an element \emph{nebular} when it is in this ideal.
\end{definition}

\begin{propdef}
  We say $\Gamma$ is \emph{faithful} when any of these equivalent statements are true:
  \begin{enumerate}
    \item $\Psi_\Gamma$ is injective
    \item $\Gamma$ has trivial nebula (meaning it is not nebulous)
    \item There are no nebular non-zero elements in $\Gamma$
  \end{enumerate}
\end{propdef}

\begin{proof}
  (2) and (3) are trivially equvialent. (2) is equivalent to the kernel of the symmetric part of $\Psi_\Gamma$ being trivial, which is again equivalent to injectivity of the symmetric part of $\Psi_\Gamma$, which finally is equivalent to injectivity of $\Psi_\Gamma$. 
\end{proof}

\begin{proposition}
  We make some observations on the nebula of $\Gamma = (R, k, \upsilon)$:
  \begin{enumerate}
    \item We have $N(\Gamma) \subseteq \ker \upsilon$. 
    \item $N(\Gamma)$ is the largest sub-ideal of $\ker \upsilon$ that is a $\lambda$-ideal. \todo{more precise notion of large}
    \item From these two, $\Gamma$ is faithful if and only if there is no non-trivial $\lambda$-ideal contained in $\ker \upsilon$.
  \end{enumerate}
\end{proposition}

\begin{proof}
  We prove these one by one.
  \begin{enumerate}
    \item It is clear that if $x \in N(\Gamma)$ then $\upsilon(\psi^1(x)) = 0$ and $x \in \ker \upsilon$.
    \item Notice that any element $x \in \ker \upsilon$ such that $\lambda^n(x) \notin \ker \upsilon$ with $n \ge 1$ would have a non-zero $\lambda$-shadow and hence not be in the nebula by definition, as all $\psi$-shadows must be zero. Similarly, if $\lambda^n(x) \in \ker \upsilon$ for all $n \ge 1$, then the $\lambda$-shadows are all zero, which imply that the $\psi$-shadows must likewise be zero. This reasoning \todo{Check this properly} uses \ref{prop:naturalCommutativity}.
    \item Suppose there were such an ideal $I$. Then by (2), $N(\Gamma)$ may not be trivial, as $I$ is larger than the trivial ideal. Also, suppose there were no such ideal. Then by (1), $N(\Gamma)$ must be trivial and $\Gamma$ faithful. 
  \end{enumerate}
\end{proof}

\begin{proposition}
  Suppose we have an umbral morphism $\phi$ with a faithful domain. Then $\phi$ must be injective. 
\end{proposition}

\begin{proof}
  Let $\phi = (f, g)$. We have from \ref{propdef:upsilonUniversalDiagram} the diagram
  \begin{center}
    \begin{tikzcd}
      \Gamma \arrow[d, hook, "\Psi_\Gamma"] \arrow[r, "\phi"] & \Gamma' \arrow[d, "\Psi_{\Gamma'}"] \\
      \Upsilon_k \arrow[r, hook, "\Upsilon_{g}"] & \Upsilon_{k'}
    \end{tikzcd}
  \end{center}
  It is clear that since $g$ is injective (it is a morphism of fields), so is $\Upsilon_{g}$. Suppose we have some non-zero element $r$ in the symmetry part of $\Gamma$ such that $f(r) = 0$. We further see that $\Psi_{\Gamma'}(f(r)) = (k \mapsto 0)$. However, by the diagram, $\Psi_{\Gamma'}(f(r)) = \Upsilon_g(\Psi_{\Gamma}(r))$ and so we get $\Upsilon_g(\Psi_{\Gamma}(r)) = (k \mapsto 0)$. Since both of these are injective we get $r = 0$, which is a contradiction. Hence $\phi$ has injective symmetry-part, and is thus injective. \qedhere
\end{proof}

\subsection{The Image of a Symmetric Shadow}
An interesting question about zeta type formulas is what kind of zeta types they give rise to. This is encapsulated in the image of a symmetric shadow. We define this concept, and make some basic observations.

\begin{definition}
  We define the image of $\Gamma$, $Im(\Gamma)$ as the image of the symmetric part of $\Psi_\Gamma$. This is thought of as the set of zeta types in $\Upsilon_k$ that come from some specific type of higher object.  
\end{definition}

\subsection{Some Simple Symmetric Shadows}
We will now review some basic symmetric shadows.

\subsection{The Rational Shadow}

\subsection{Tannakian Shadows}
