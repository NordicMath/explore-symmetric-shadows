\section{Symmetric Shadows}

In general, a zeta type is thought of as some kind of numerical manifestation of a higher object, or a \emph{shadow}, if you will. With this intuition in mind it makes sense to form axioms for structures that give rise to zeta types, which we informally call zeta type formulas.
 
 The trivial approach would be to define this as a function $f : A \to \Upsilon_k$, where $A$ would be the set of higher objects of a particular kind, and $k$ is a field containing the kind of numbers that the objects give rise to (for instance the rationals containing integers). 
 
 However, this ignores all structural aspects of zeta type formulas, which at the very least in some sense should respect addition and product (a central philosophy of zeta types is that they should simplify complicated structures faithfully). This can be encapsulated by making $f$ a ring-homomorphism and enriching $A$ with some kind of addition and multiplication. 
 
 Further, something truly beatiful happens if we require that the zeta types formula also respect some notion of symmetry. $\lambda$-rings serve as precise algebraic notion of symmetry, and hence we make $f$ a $\lambda$-homomorphism and define suitable $\lambda$-operations on $A$. What now happens is that the formula $f$ collapses into a single ring-homomorphism $\upsilon : A \to k$, which is both sufficient and required for the existence of a unique formula $f : A \to \Upsilon_k$.

\subsection{Central definition and the Umbra}
We begin by defining the central concepts of symmetric shadows and some terminology, as well as establishing the relation between zeta type formulas and ring-homomorphisms to the base field. 


\begin{definition}
  A \emph{(local) symmetric shadow} is a triple $(R, k, \upsilon)$ where $R$ is a $\lambda$-ring called the \emph{symmetry realm}, $k$ is a field called the \emph{shadow field}, and $\upsilon: R \to k$ is a ring-homomorphism called the \emph{umbra}. 
  \begin{enumerate}
    \item An \emph{element} of the symmetric shadow refers to an element of the symmetry realm. 
    \item A symmetric shadow \emph{over} $k$ has $k$ as shadow field, and one \emph{under} $R$ has $R$ as symmetry realm.
  \end{enumerate}
\end{definition}

\subsection{Morphisms of Symmetric Shadows}
The notion of $f_\Gamma$ can be slightly cleaned up by defining morphisms between symmetric shadows. We also establish some more terminology, and reintroduce $\Upsilon_k$ as a symmetric shadow itself. 


\subsection{The Nebula}
An ideal situation for zeta type formulas is that they should be faithful, ie. different objects should have different zeta types. We introduce the nebula to measure just how faithful a symmetric shadow is. We also make central observations that can be used to compute the nebula, and show some of the interesting consequences of faithfulness. 


\subsection{The Image of a Symmetric Shadow}
An interesting question about zeta type formulas is what kind of zeta types they give rise to. This is encapsulated in the image of a symmetric shadow. We define this concept, and make some basic observations.

\subsection{Some Simple Symmetric Shadows}
We will now review some basic symmetric shadows.

\subsection{The Rational Shadow}

\subsection{Tannakian Shadows}
